\input{preamble.txt}
\usepackage{multicol}

\begin{document}
	\section{Version Control}

	\begin{itemize}
		\item Two flavours of Version Control
		\begin{itemize}
			\item Centralized (B07 uses this)
			\item Decentralized
		\end{itemize}

		\item Centralised Version
		\begin{itemize}
			\item Keep code in a centralized location (the “Repository”)
			\item Code in repository is the ``Master Copy" (\textbf{\underline{Never}} directly modify)
			\item Instead make local copies of the repository on each computer you will be working on (working copy)
			\item When major changes are made to local copy that you want to save, ``commit" change to repo
			\item Tools allow you to revert to a previous version of the source code (only when commits have occurred)
		\end{itemize}

		\item Some Terminology
%		\begin{minipage}[t][0.72in]{0.25\textwidth}
%			\begin{itemize}
%				\item Repository/Repo
%				\item Client program
%				\item Working copy
%			\end{itemize}
%		\end{minipage}
%		\begin{minipage}[t]{0.4\textwidth}
%			\begin{itemize}
%				\item Checkout
%				\item Commit
%			\end{itemize}
%		\end{minipage}
		\begin{multicols}{4}
			\begin{itemize}
				\item Repository/Repo
				\item Client program
				\item Working copy
				\item Checkout
				\item Commit
				\item[\phantom{Spacekeeper}]
			\end{itemize}
		\end{multicols}


		\item Centralized systems include:
		\begin{itemize}
			\item \textbf{S}ub\textbf{V}ersio\textbf{n} (\textbf{SVN})
			\begin{itemize}
				\item SVN is the successor to \textbf{C}oncurrent \textbf{V}ersions \textbf{S}ystem (\textbf{CVS}), and was built to help fix many issues in CVS
			\end{itemize}
			\begin{multicols}{4}
				\item Git
				\item Mercurial
				\item ClearCase
				\item Perforce
			\end{multicols}
		\end{itemize}

		\item SSH and SCP are \textbf{not} version control systems
		\begin{itemize}
			\item \textbf{S}ecure \textbf{Sh}ell (SSH) is used to connect to a remote computer and work in a shell on that computer
			\item \textbf{S}ecure \textbf{C}o\textbf{p}y (SCP) is used to:
			\begin{itemize}
				\item Securely copy files from one computer to another
				\item Transfer a copy of the files but does \textbf{not} version them
			\end{itemize}
		\end{itemize}

		\item Version Control – Managing Concurrency

		\emph{When two or more people want to edit the same file at the same time}
		\begin{itemize}
			\item Pessimistic concurrency
			\begin{itemize}
				\item Only allow one writeable copy of each file
				\item e.g. Microsoft Visual SourceSafe, Rational ClearCase
			\end{itemize}

			\item Optimistic concurrency
			\begin{itemize}
				\item Allow writes, fix issues afterwards
				\item Merging
				\begin{itemize}
					\item SVN is either able to merge without help from the user, or
					\item \textit{Conflict}: SVN needs the user to resolve the conflict
				\end{itemize}
				\item e.g. Subversion, CVS, Perforce
			\end{itemize}
			\newpage
			\item Optimistic Concurrency – Merging Options

			Select from: \textbf{(p)} postpone, \textbf{(df)} diff-full, \textbf{(e)} edit, \textbf{(mc)} mine-conflict, \textbf{(tc)} theirs-conflict and \textbf{(s)} show all options.\\[-15pt]
			\begin{center}
				\begin{minipage}[c]{0.6\textwidth}
					\begin{itemize}
						\item[(e) edit] - changed merged file in an editor
						\item[(df) diff-full] - show all changes made to merged file4
						\item[(r) resolved]  - accept merged version of file\\
						\item[(dc) display-conflict] - show all conflicts (ignoring merged version)
						\item[(mc) mine-conflict] -  accept my version for all conflicts (same as above)
						\item[(tc) theirs-conflict] -  accept their version for all conflicts (same as above) \\
						\item[(mf) mine-full] - accept my version of entire file (even non-conflicts)
						\item[(tf) theirs-full]  - accept my version of entire file (same as above)\\
						\item[(p) postpone] - mark the conflict to be stored later
						\item[(l) launch] - launch external tool to resolve conflict
						\item[(s) show all] - show this list
					\end{itemize}
				\end{minipage}
			\end{center}
		\end{itemize}

		\item Integrating the code - Reasons for merge conflicts\\[-15pt]

		\begin{minipage}[t]{0.3\textwidth}
			\begin{itemize}
				\item Communication
				\item Complex code bases
				\item Experimental features being built
			\end{itemize}
		\end{minipage}
		\begin{minipage}[t]{0.5\textwidth}
			\begin{itemize}
				\item More than one project on the go that impacts this code
				\item Two features being built in same class by different developers
			\end{itemize}
		\end{minipage}

		\item Branching
		\begin{itemize}
			\item Branches are divergent copies of development lines
			\item These versions are used to build out complex features, or do experiments,
			without having an impact on the main code line
			\item Strategies include:
			\begin{itemize}
				\item No branching
				\item Release branching
				\item Feature branching
			\end{itemize}
		\end{itemize}

		\item Storage scheme
		\begin{itemize}
			\item Storing every copy of every file generated over the course of a project is not practical
			\item Version control systems store incremental differences in files/folder structures
			\item These differences store enough information to re-construct previous versions, without storing every single copy ever made of the file
		\end{itemize}

		\item What’s Stored Where
		\begin{itemize}
			\item Server side: out of scope
			\item Local copy contains a special directory, \textbf{.svn}
			\begin{itemize}
				\item It stores (locally) the information subversion needs to keep track of your files, version numbers, where the repository is, etc.
				\item Needless to say, you should not mess with the contents of this directory. Let subversion do its job
			\end{itemize}
		\end{itemize}

		\item General rules
		\begin{itemize}
			\item Update and commit frequently
			\item Never break the main branch
			\item Always comment clearly what changes are in a revision
			\item Test all code before accepting merge
			\item Communicate with your team!
		\end{itemize}
	\end{itemize}
\end{document}